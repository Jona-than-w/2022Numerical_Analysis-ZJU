\documentclass[UTF8]{ctexart}
\usepackage{setspace}
\usepackage[letterpaper,top=2cm,bottom=2cm,left=3cm,right=3cm,marginparwidth=1.75cm]{geometry}
\CTEXsetup[format={\Large\bfseries}]{section}
\usepackage{amsmath}
\usepackage{mathabx}
\usepackage[mathscr]{eucal}

\title{Numerical Analysis HW2}

\author{数学与应用数学2002 王锦宸 }
\date{October 2022}

\begin{document}

\maketitle

\section{}
\subsection{Determine $\xi (x)$ explicitly}
$$l_1(x) = \frac{x-2}{1-2},\quad l_2(x) = \frac{x-1}{2-1}$$
$$p_1(f;x) = 1* \frac {x-2}{1-2} + \frac{1}{2}*\frac{x-1}{2-1}$$
Besides,\\
$$f''(x) = \frac{2}{x^3}$$
Thus,\\
$$\frac{1}{x} - (-\frac{x}{2} + \frac{3}{2})= \frac{1}{(\xi (x))^3} (x-1)(x-2)$$
$$\xi(x) = \sqrt[3]{2x}$$
\subsection{Extend the domain of $\xi$ continuously from $(x_0,x_1)$ to $[x_0,x_1]$.Find max $\xi(x)$,\; min$\xi (x)$, and max $f''(\xi (x))$.}
\noindent Since $\xi (x)$ is monotonically increasing on $[1,2]$,
$$max\:\xi (x) = \xi (2) = \sqrt[3]{4},\quad min\:\xi (x) = \xi (1) = \sqrt[3]{2}$$
Since $f''(\xi(x)) = \frac{1}{x}\;$ is monotonically decreasing on $[1,2]$, 
$$max\:f''(\xi(x)) = f''(\xi(1)) = 1$$

\section{}
Since $f_i > 0$ for i = 0,1,\dots,n, let $p_0(x_i) = \sqrt{f_i}$ for i = 0,1,\dots,n. By the interpolation theorem, we can uniquely determine a polynomial of degree $\leqq n$. Let $ p(x) = (p_0(x))^2$, p(x) is a polynomial of degree $\leqq 2n$ that are non-negative on the real line and $p(x_i) = f_i$ for i = 0,1,\dots,n.

\section{}
\subsection{Prove by introduction}
\noindent when n = 1,
$$f[t,t+1] = e^{t+1} - e^t = \frac{(e-1)^1}{1!}e^t$$
Assume that (when n = k),
$$f[t,t+1,\dots,t+k] = \frac{(e-1)^k}{k!}e^t$$
when n = k+1,
\begin{equation}
    \begin{aligned}
        f[t,t+1,\dots,t+k+1] &= \frac{f[t+1,t+2,\dots,t+k+1] - f[t,t+1,\dots,t+k]}{t+k+1 - t}\\
        &= \frac{\frac{(e-1)^{k}}{k!}e^{t+1} - \frac{(e-1)^k}{k!}e^t}{k+1}\\
        &= \frac{\frac{(e-1)^{k+1}}{k!}e^{t} + \frac{(e-1)^{k}}{k!}e^{t} - \frac{(e-1)^k}{k!}e^t}{k+1}\\
        &= \frac{(e-1)^{k+1}}{(k+1)!}e^{t}\\
    \nonumber
    \end{aligned}
\end{equation}
By induction,
$$\forall x \in \mathbb{R} , \; f[t,t+1,\dots,t+k+1] = \frac{(e-1)^{k+1}}{(k+1)!}e^{t}$$

\subsection{Is $\xi$ located to the left or to the right of the midpoint?}
\noindent From 3.1,
$$f[0,1,\dots,n] = \frac{(e-1)^{n}}{n!}e^{0} = \frac{(e-1)^{n}}{n!}.$$
From Corollary 2.2 we know,
$$\exists \; \xi \in (0,n) \quad s.t. \quad f[0,1,\dots,n] = \frac{1}{n!}f^{(n)}(\xi).$$
Thus,
\begin{equation}
    \begin{aligned}
            f^{(n)}(\xi) &= (e-1)^{n}.\\
            f(x) &= e^(x).\\
            f^{(n)}(x) &= e^{x}.\\
            e^{\xi} &= (e-1)^n\\
            \xi &= n \ln (e - 1)\\
       \nonumber 
    \end{aligned}
\end{equation}
Since $n \ln (e - 1) \geqq \frac{n}{2}$,\quad$\xi$ is located to the right of the midpoint.

\section{}
\subsection{Use the Newton formula to obtain $p_3(f;x)$}
Since we have,
\begin{equation}
    \begin{array}{ccccc}
    x & 0 & 1 & 3 & 4 \\
    f(x) & 5 & 3 & 5 & 12
    \nonumber
    \end{array}
\end{equation}
we can construct the following table of divided difference,
\begin{equation}
    \begin{tabular}{c|cccc}
    0 & 5 & & & \\
    1 & 3 &  -2  & & \\
    3 & 5 & 1 & 1 & \\
    4 & 12 & 7 & 2 &  1/4 
    \nonumber
    \end{tabular}
\end{equation}
Thus,
$$p_{3}(f ; x)=5-2 x+x(x-1)+\frac{1}{4} x(x-1)(x-3)=\frac{1}{4} x^{3}-\frac{9}{4} x+5$$

\subsection{Find an approximate value for the location $x_{min}$ of the minimum.}
\noindent Let  $p_{3}^{\prime}=\frac{3}{4} x^{2}-\frac{9}{4}=0$ , then  $x=\pm \sqrt{3}$.\\
Then when  $x \in(1, \sqrt{3})$, p  is monotonic decreasing, while  $x \in(\sqrt{3}, 3)$, p  is increasing. That is, $f_{\min }=p(\sqrt{3})=5-\frac{3 \sqrt{3}}{2} \approx 2.402$ and $x_{min } \approx 1.732$. 

\section{}
\subsection{Compute f [0,1,1,1,2,2]}
Since  $f=x^{7}$ , then the table of divided differences can be established as follow,
\begin{equation}
\begin{tabular}{c|cccccc}
0 & 0 & & & & & \\
1 & 1 & 1 & & & & \\
1 & 1 & 7 & 6 & & & \\
1 & 1 & 7 & 21 & 15 & & \\
2 & 128 & 127 & 120 & 99 & 42 & \\
2 & 128 & 448 & 321 & 201 & 102 & 30
\nonumber
\end{tabular}
\end{equation}
Thus, f[0,1,1,1,2,2] = 30 
\subsection{Determine $\xi$}
Since $f^{(5)}(x)=2520x^{2}=30$ , then  $x=\frac{1}{2 \sqrt{21}} \approx 0.1091$. 

\section{}
\subsection{Estimate f(2) using Hermite interpolation.}
We can obtain the table of divided differences,
\begin{equation}
    \begin{tabular}{c|ccccc}
    0 & 1 & & & & \\
    1 & 2 & 1 & & & \\ 
    1 & 2 & -1 & -2 & & \\
    3 & 0 & -1 & 0 & 2 / 3 & \\
    3 & 0 & 0 & 1 / 2 & 1 / 4 & -5/36\\
    \nonumber
    \end{tabular}
\end{equation}
Thus,
$$p(x)=1+x-2 x(x-1)+\frac{2}{3} x(x-1)^{2}-\frac{5}{36} x(x-1)^{2}(x-3)$$
$$f(2) \approx p(2)=\frac{11}{18}$$

\subsection{Estimate the maximum possible error of the above answer.}
Since N = 2 + 1 + 1 = 4, by Theorem 2.35, we have,
\begin{equation}
    \begin{aligned}
    f(x)-p_{5}(f ; x) &=\frac{f^{(N+1)}(\xi)}{(N+1) !} \prod_{i=0}^{k}\left(x-x_{i}\right)^{m_{i}+1} \\
    &=\frac{f^{(5)}(\xi)}{5 !} x(x-1)^{2}(x-3)^{2}\\
    \nonumber
    \end{aligned}
\end{equation}
Thus, $|f(2) - p(2)| \leqq \frac{M}{60}$.

\section{}
Let's prove it by induction.\\
When k = 1,
\begin{equation}
    \begin{aligned}
         \bigtriangleup ^1 f(x) &= f(x+h) - f(x)\\
         &= 1!h^1f[x,x+h]\\
         \nonumber
    \end{aligned}
\end{equation}
Assume that(when k = n),
$$\bigtriangleup f(x) = k!h^kf[x_0,x_1,\dots ,x_k]$$
when k = n + 1,
\begin{equation}
    \begin{aligned}
        \bigtriangleup ^{k+1} f(x) &= \bigtriangleup ^{k} f(x+h) - \bigtriangleup ^{k} f(x)\\
        &= k!h^kf[x_1,x_2,\dots ,x_{k+1}] - k!h^kf[x_0,x_1,\dots ,x_k]\\
        &= k!h^k(f[x_1,x_2,\dots ,x_{k+1}] - f[x_0,x_1,\dots ,x_{k}]\\
        &= k!h^k(f[x_0,x_1,\dots ,x_{k+1}])(x_{k+1} - x_0)\\
        &= k!h^k(f[x_0,x_1,\dots ,x_{k+1}])(k+1)h\\
        &= (k+1)!h^{k+1}(f[x_0,x_1,\dots ,x_{k+1}])
        \nonumber
    \end{aligned}
\end{equation}
The proof for backward difference $\bigtriangledown$ is similar.

\section{}
Let's prove by induction.\\
When n = 1,\\
\begin{equation}
    \begin{aligned}
        \frac{\partial}{\partial x_0}f[x_0,x_1] &= \frac{\partial}{\partial x_0}\left( \frac{f(x_0) - f(x_1)}{x_0 - x_1} \right)\\
        &= \frac{f'(x_0)(x_0-x_1) - (f(x_0) - f(x_1))}{(x_0 - x_1)^2}\\
        &= \frac{f[x_0,x_1] - f'(x_0)}{x_1 - x_0}\\
        & = f[x_0,x_0,x_1]
    \nonumber
    \end{aligned}
\end{equation}
Assume that it holds for n = k,\\
When n = k + 1,
\begin{equation}
    \begin{aligned}
         \frac{\partial}{\partial x_0}f[x_0,x_1,\dots , x_{k+1}] &=  \frac{\partial}{\partial x_0}\left( \frac{f[x_1,\dots , x_{k+1}] - f[x_0,x_1,\dots,x_k]}{x_{k+1} - x_0} \right)\\
         &= \frac{-f[x_0,x_0,x_1,\dots,x_k](x_{k+1}-x_0) + (f[x_1,\dots , x_{k+1}] - f[x_0,x_1,\dots,x_k])}{(x_{k+1} - x_0)^2}\\
         &= \frac{f[x_0,x_1,\dots,x_{k+1}] - f[x_0,x_0,x_1,\dots,x_k]}{x_{k+1} - x_0}\\
         &= f[x_0,x_0,x_1,\dots,x_{k+1}]\\
    \nonumber
    \end{aligned}
\end{equation}

\section{A min-max problem}
\noindent Let $t = \frac{2x - (a+b)}{b-a}\quad (t \in [-1,1])$, that is, $x = \frac{t(b-a)+(a+b)}{2}$.\\
Thus, we have $q(t) = p\left( \frac{t(b-a)+(a+b))}{2}\right) = p(x)$, in which the coefficient of $t^n$ is $a_0 \left( \frac{b-a}{2} \right)^n$.\\
By Theorem 2.44(Chebyshev),
$$\forall \; q \in \tilde{\mathbb{P}_n},\quad max_{t \in [-1,1]} \left| \frac{q(t)}{a_0 \left( \frac{b-a}{2} \right)^n}\right| \geqq max_{t \in [-1,1]} \left| \frac{T_n(t)}{2^{n-1}}\right| $$

that is,
$$min \: max_{t \in [-1,1]} \left| \frac{q(t)}{a_0 \left( \frac{b-a}{2} \right)^n}\right| = max_{t \in [-1,1}] \left| \frac{T_n(t)}{2^{n-1}}\right|$$
Therefore,
\begin{equation}
    \begin{aligned}
        min \: max_x \in [a,b] p(x) &= min \: max_t \in [-1,1] q(t)\\
        &= \frac{1}{2^n}a_0\left(\frac{}{}\right)^n\\
        &= a_0\frac{(b-a)^n}{2^{2n-1}}\\
    \nonumber
    \end{aligned}
\end{equation}

\section{}
\noindent First, we know $||\hat{p_n}||_{\infty} = \frac{1}{T_n(a)}$\\
By the property of $T_n$ we have,
$$\hat{p}_{n}(x)\left(x_{k}^{\prime}\right)=\frac{(-1)^{k}}{T_{n}(a)} \quad \text { for } \quad x_{k}^{\prime}=\cos \frac{k}{n} \pi, k=0,1, \ldots, n$$
Suppose that $\exists \quad p \in \mathcal{P}_n^a,\quad s.t. \quad ||p||_{\infty} <  \frac{1}{|T_n(a)|}$.\\
Consider the polynomial $Q(x) = \frac{1}{|T_n(a)|}T_n(x) - p(x)$.
$$Q(x_k') = \frac{(-1)^k}{|T_n(a)|} - p(x_{k'}),\quad k = 0,1,\dots,n.$$
Obviously, Q(x) has alternating signs at $x'_0,x'_1,\dots,x'_n$. Hence Q(x) must have n zeros. However, by the construction of Q(x), the degree of Q(x) is at most n - 1. Therefore, 
$Q(x) \equiv 0$, that is, $||p||_{\infty} = \frac{1}{|T_n(a)|}$, which is contradict to the assumption.\\
Therefore,
$$\forall p \in \mathbb{P}_{n}^{a}, \quad \left\| \hat{p}_{n}\right\| _{\infty} \leq \|p\|_{\infty}$$

\section{Prove Lemma 2.48}
\subsection*{2.50(a)}
Since $t \in (0,1)$, every factor of this polynomial is positive. Hence it holds.

\subsection*{2.50)b)}
By the Binomial Theorem,
$$1=(t+(1-t))^{n}=\sum_{k=0}^{n}\left(\begin{array}{l}
n \\
k
\end{array}\right) t^{k}(1-t)^{n-k}=\sum_{k=0}^{n} b_{n, k}(t)$$

\subsection*{2.50(c)}
Derive on both sides of the equation below,
$$(p+q)^{n}=\sum_{k=0}^{n}\left(\begin{array}{l}
n \\
k
\end{array}\right) p^{k} q^{n-k}$$
we have,
$$n(p+q)^{n-1}=\sum_{k=0}^{n}\left(\begin{array}{l}
n \\
k
\end{array}\right) k p^{k-1} q^{n-k}$$
Multiple both sides p times, we have,
$$n p(p+q)^{n-1}=\sum_{k=0}^{n}\left(\begin{array}{l}
n \\
k
\end{array}\right) k p^{k} q^{n-k}$$
Then we let p = t and q = 1 - t, we have, 
$$n p=\sum_{k=0}^{n}\left(\begin{array}{l}
n \\
k
\end{array}\right) k t^{k}(1-t)^{n-k}=\sum_{k=0}^{n} k b_{n, k}(t)$$

\subsection*{2.50(d)}
Take derivative on the both sides of the following equation,
$$n p(p+q)^{n-1}=\sum_{k=0}^{n}\left(\begin{array}{l}
n \\
k
\end{array}\right) k p^{k} q^{n-k}$$
We have,
$$n(p+q)^{n-1}+n(n-1) p(p+q)^{n-2}=\sum_{k=0}^{n}\left(\begin{array}{l}
n \\
k
\end{array}\right) k^{2} p^{k-1} q^{n-k}$$
Multiple both sides p times, we have,
$$n p(p+q)^{n-1}+n(n-1) p^{2}(p+q)^{n-2}=\sum_{k=0}^{n}\left(\begin{array}{l}
n \\
k
\end{array}\right) k^{2} p^{k} q^{n-k}$$
let p = t and q = 1 - t, we have,
$$n t+n(n-1) t^{2}=\sum_{k=0}^{n} k^{2} b_{n, k}(t)$$
By the result of 2.50(b) and 2.50(c), we have,
\begin{equation}
    \begin{aligned}
    \sum_{k=0}^{n}(k-n t)^{2} b_{n, k}(t) &=\sum_{k=0}^{n} k^{2} b_{n, k}(t)-2 n t \sum_{k=0}^{n} k b_{n, k}(t)+(n t)^{2} \sum_{k=0}^{n} b_{n, k}(t) \\
    &=n t+n(n-1) t^{2}-2(n t)^{2}+(n t)^{2}=n t-n t^{2}=n t(1-t)
    \nonumber
\end{aligned}
\end{equation}









\end{document}
